\documentclass[a4paper,10pt]{article}
\usepackage[utf8]{inputenc}
\usepackage[french]{babel}
\usepackage[T1]{fontenc}
\usepackage[utf8]{inputenc}
\usepackage{fullpage}
% Pour les checkmarks
\usepackage{pifont}
\usepackage{amssymb}
% Pour des \hline en gras
\usepackage{makecell}

\newcommand{\hlineGras}{\Xhline{2\arrayrulewidth}}

\title {Fiche de suivi - projet de statistiques\\ Estimateur Jackknife}
\author{Manon Ansart\\ Antoine Augusti}
\date{16 mai 2014}

\begin{document}

	\maketitle

	\section{Présentation du sujet}
	Nous avons choisi lors de ce projet de statistiques de travailler sur l'estimateur Jackknife. L'estimateur Jackknife permet d'estimer la précision d'un échantillon statistique : d'une médiane, de la variance, d'un quantile. Pour cela, il utilise des sous-ensembles des données disponibles observées.

	\section{Plan du projet}
	Retrouvez ci-dessous l'approche que nous avons choisi de prendre pour présenter notre projet.

	\vspace{10px}
	\begin{enumerate}
		\item Présentation : principe du ré-échantillonnage
		\begin{enumerate}
			\item Origine
			\item Principe
		\end{enumerate}
		\item Application au biais
		\begin{enumerate}
			\item Estimation Jackknife du biais
			\item Réduction du biais
		\end{enumerate}
		\item Application à l'évaluation de la stabilité
		\begin{enumerate}
			\item Approche intuitive
			\item Décision à propos de la stabilité (exemple)
		\end{enumerate}
		\item Autres applications
		\begin{enumerate}
			\item Autres estimations
			\item Autres méthodes
		\end{enumerate}
	\end{enumerate}


	\section{Répartition du travail}
	Retrouvez dans le tableau ci-dessous comment nous avons réparti entre nous les différentes parties de ce projet.

	\vspace{10px}
	\begin{center}
		\begin{tabular}
		{| l || c | c |} \hline
		Partie & Manon Ansart & Antoine Augusti \\ \hline \hline
		\textbf{Présentation} & & \\ \hline
		Origine & \checkmark & \checkmark \\ \hline
		Principe & \checkmark & \checkmark \\ \hline
		\hlineGras
		\textbf{Application au biais} & & \\ \hline
		Estimation Jackknife du biais  & & \checkmark  \\ \hline
		Réduction du biais & \checkmark &  \\ \hline
		\hlineGras
		\textbf{Application à l'évaluation de la stabilité} & & \\ \hline
		Approche intuitive  & & \checkmark \\ \hline
		Décision à propos de la stabilité (exemple)  & & \checkmark \\ \hline
		\hlineGras
		\textbf{Autres applications} & & \\ \hline
		Autres estimations  & \checkmark & \\ \hline
		Autres méthodes  & \checkmark & \\ \hline
		\end{tabular}
	\end{center}

\end{document}