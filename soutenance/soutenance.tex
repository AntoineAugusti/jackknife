% L'option handout permet de supprimer la barre de navigation
\documentclass[handout]{beamer}
\usepackage[utf8]{inputenc}
\usepackage[french]{babel}
\usepackage[T1]{fontenc}
\usepackage{amsmath}
% Pour pouvoir insérer des images
\usepackage{graphicx}
\usepackage{wrapfig}
\graphicspath{images/}
% Gestion des couleurs
\usepackage{color}
\definecolor{red}{RGB}{231, 76, 60}

% Un joli thème flat
\usetheme{Rochester}

% Personnalisation du thème
\usecolortheme[named=red]{structure}
% Numéro de slides dans le footer
\setbeamertemplate{footline}[frame number]
\setbeamertemplate{blocks}[shadow=false]

% ------------------------------------ %
% -- METADONNÉES DU DOCUMENT --------- %
\title{
	Estimateur Jackknife
}
\author{
	Manon \textsc{Ansart} \\
	\vspace{10px}
	Antoine \textsc{Augusti}
}
\date{}

% Début du document
\begin{document}

	% Génération de la page de titre
	\begin{frame}[plain]
		\titlepage
	\end{frame}

	% Génération du sommaire
	\begin{frame}[plain]
		\frametitle{Sommaire}
		\tableofcontents
	\end{frame}


	% //////////////////////////////// %
	% /// Présentation /////////////// %
	\section{Présentation}

		\subsection{Le ré-échantillonnage}
		\begin{frame}
			\frametitle{Le ré-échantillonnage}

			L'objectif de l'estimateur Jackknife :
			\begin{itemize}
				\item estimer la précision d'un échantillon statistique : d'une médiane, de la variance, d'un quantile\dots
			\end{itemize}

			\vspace{15px}

			La méthode utilisée :
			\begin{itemize}
				\item utilisation de sous-ensembles des données disponibles observées
			\end{itemize}

			\vspace{20px}

			\begin{exampleblock}{Bootstrap}
				Même objectif que le Jackknife mais utilisation d'un tirage aléatoire avec remise à partir de l'ensemble des données.
			\end{exampleblock}
		\end{frame}

	% ////////////////////////////////////////////// %
	% /// Biais de l'estimateur de la moyenne ////// %
	\section{Biais de l'estimateur de la moyenne}
		\subsection{Biais de l'estimateur de la moyenne}
			\begin{frame}
			\frametitle{Biais de l'estimateur de la moyenne}
			Soit $n$ observations indépendantes notées $x_1, \dots, x_n$ provenant d'une loi inconnue $X$ ayant une caractéristique $\theta$ que l'on souhaite estimer. Soit $T$ un estimateur de $\theta$. On souhaite connaître le biais et l'écart type de $T$.

			\begin{equation}
				E(T) = \theta + \frac{a_{1}(\theta)}{n} + \frac{a_{2}(\theta)}{n^2} + \dots
			\end{equation}

			\textbf{Écart des moyennes réduites}
			\begin{equation}
				I_j = (n - 1)(T - T_{-j})
			\end{equation}
			où $T_{-j}$ est l'estimateur $T$ calculé sur $X$ sans l'échantillon $x_j$.

			\vspace{5px}
			\textbf{Biais principal de l'estimateur de la moyenne}
			\begin{equation}
				\overline{I} = \frac{1}{n} \sum\limits_{j=1}^n I_j = \frac{a_{1}(\theta)}{n}
			\end{equation}

		\end{frame}

		\subsection{Exemple : biais de l'estimateur de la moyenne}
			\begin{frame}
			\frametitle{Exemple : biais de l'estimateur de la moyenne}
			Soit $X = \{1, 2, 3, 4\}$ donc $n = 4$. On pose $T = \frac{1}{n} \sum\limits_{i=0}^n x_i  = \hat{\theta} = 2.5$

			On a alors :\\
			$I_1 = (n - 1)(T - T_{-1}) = 3 (2.5 - 3) = -1.5$\\
			$I_2 = 3 (2.5 - 2.66) = -0.48$\\
			$I_3 = 3 (2.5 - 2.33) = 0.51$\\
			$I_4 = 3 (2.5 - 2) = 1.5$\\

			\vspace{5px}
			et :\\
			$\overline{I} = \frac{1}{n} \sum\limits_{j=1}^n I_j = \frac{1}{4} (-1.5 + -0.48 + 0.51 + 1.5) = \frac{3}{400}$

			\vspace{5px}
			la forme ajustée de $T$ avec biais est alors :\\
			\[ \tilde{T} = T + \overline{I} = 2.5 + \frac{3}{400} = 2.5075 \]

		\end{frame}
% Fin du document
\end{document}