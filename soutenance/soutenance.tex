% L'option handout permet de supprimer la barre de navigation
\documentclass[handout]{beamer}
\usepackage[utf8]{inputenc}
\usepackage[french]{babel}
\usepackage[T1]{fontenc}
\usepackage{amsmath}
% Pour pouvoir insérer des images
\usepackage{graphicx}
\usepackage{wrapfig}
\graphicspath{images/}
% Gestion des couleurs
\usepackage{color}
\definecolor{red}{RGB}{231, 76, 60}

% Un joli thème flat
\usetheme{Rochester}

% Personnalisation du thème
\usecolortheme[named=red]{structure}
% Numéro de slides dans le footer
\setbeamertemplate{footline}[frame number]
\setbeamertemplate{blocks}[shadow=false]

\newcommand{\intervalleoo}[2]{\mathopen{]}#1\,;#2\mathclose{[}}
\newcommand{\intervalleff}[2]{\mathopen{[}#1\,;#2\mathclose{]}}
\newcommand{\intervalleof}[2]{\mathopen{]}#1\,;#2\mathclose{]}}
\newcommand{\intervallefo}[2]{\mathopen{[}#1\,;#2\mathclose{[}}
\newcommand{\intervalle}[2]{\mathopen{(}#1\,;#2\mathclose{)}}

% ------------------------------------ %
% -- METADONNÉES DU DOCUMENT --------- %
\title{
	Estimateur Jackknife
}
\author{
	Manon \textsc{Ansart} \\
	\vspace{10px}
	Antoine \textsc{Augusti}
}
\date{19 mai 2014}

% Générer une page de titre à chaque début de section
\AtBeginSection[]
{
	\begin{frame}[plain]
	\frametitle{Sommaire}
	\tableofcontents[currentsection, hideothersubsections]
	\end{frame}
}

% Début du document
\begin{document}

	% Génération de la page de titre
	\begin{frame}[plain]
		\titlepage
	\end{frame}

	% Génération du sommaire
	\begin{frame}[plain]
		\frametitle{Sommaire}
		\tableofcontents
	\end{frame}


	% //////////////////////////////// %
	% /// Présentation /////////////// %
	\section{Présentation}

		%%
		%%% Origine
		%%
		\subsection{Origine}
		\begin{frame}
			\frametitle{Origine}
			
			Création :
			\begin{itemize}
				\item Proposé par Quenouille dans les années 50 pour le calcul de biais
				\item Repris par Tuckey pour l'estimation de la variance, tester une hypothèse ou établir un intervalle de confiance
			\end{itemize}


			\vspace{20px}

			\begin{exampleblock}{Que signifie Jackknife ?}
				Jackknife signifie en anglais couteau suisse. Cet outil a été nommé ainsi car, comme l'objet, c'est un outil pratique et rapide pouvant remplacer des outils plus sophistiqués et compliqués.
			\end{exampleblock}
		\end{frame}
		
	
		%%
		%%% Principe
		%%
		\subsection{Principe}
		\begin{frame}
			\frametitle{Principe}
% 
% 			L'objectif de l'estimateur Jackknife :
% 			\begin{itemize}
% 				\item estimer la précision d'un échantillon statistique : d'une médiane, de la variance, d'un quantile\dots
% 			\end{itemize}
% 
% 			\vspace{15px}

			La méthode utilisée :
			\begin{itemize}
				\item On souhaite estimer un paramètre $\theta$ sur un echantillon composé de n observations par un estimateur $\hat{\theta} = f(x_1,...x_i,...x_n)$
				\item Calcul de l'estimateur sur l'échantillon privé d'une observation $x_i$ : $\hat{\theta_i} = f(x_1,...x_{i-1},x_{i+1},...x_n)$
				\item Calcul de la pseudo-valeur Jackknife : $\hat{\theta_i}^* = n\hat{\theta} - (n-1) \hat{\theta_i}$
			\end{itemize}
			
			\vspace{15px}
			
			On répète ces deux dernière étapes pour chaque observation de l'échantillon. Les valeurs obtenues sont ensuite utilisées pour estimer la précision d'un échantillon statistique ou calculer le biais d'un estimateur par exemple.

% 			\vspace{20px}
% 
% 			\begin{exampleblock}{Bootstrap}
% 				Même objectif que le Jackknife mais utilisation d'un tirage aléatoire avec remise à partir de l'ensemble des données.
% 			\end{exampleblock}
		\end{frame}

	% //////////////////////////////// %
	% /// Estimation Jackknife ////// %
	\section{Application au biais}

		%%
		%%% Estimation Jackknife du biais
		%%
		\subsection{Estimation Jackknife du biais}
		\begin{frame}
			\frametitle{Estimation Jackknife du biais}
			Soit $n$ observations indépendantes notées $x_1, \dots, x_n$ provenant d'une loi inconnue $X$ ayant une caractéristique $\theta$ que l'on souhaite estimer. Soit $\hat{\theta}$ un estimateur de $\theta$.

			\vspace{15px}
			Méthode :
			\begin{itemize}
				\item Calcul des $n$ estimateurs sur l'échantillon privé d'une observation $x_i$ : $\hat{\theta_i} = f(x_1,...x_{i-1},x_{i+1},...x_n)$
				\item Calcul du biais : $biais_{jack} = (n - 1)((\frac{1}{n} \sum\limits_{i=1}^n \hat{\theta_i}) - \hat{\theta})$
			\end{itemize}


		\end{frame}

		\begin{frame}
			\frametitle{Estimation Jackknife du biais}
			Si $\hat{\theta} = \overline{x} = \frac{1}{n} \sum\limits_{i=1}^n x_i \Leftrightarrow \frac{1}{n} \sum\limits_{i=1}^n \hat{\theta_i} = \hat{\theta}$, l'estimateur est sans biais.

			\vspace{15px}
			On utilise alors l'estimateur de la variance :
			\[\hat{\sigma^2} = \frac{\sum\limits_{i=1}^n (x_i - \overline{x})^2}{n} \]

			et l'estimateur Jackknife du biais devient alors :
			\[biais_{jack} = \frac{-\hat{\sigma^2}}{n} \]
		\end{frame}
		
		%%
		%%% Estimation Jackknife du biais
		%%
		\subsection{Réduction du biais}		
		\begin{frame}
			\frametitle{Réduction du biais}
			Après avoir calculé le biais, on peut chercher à le réduire en calculant l'estimateur Jackknife.
			\vspace{15px}
			Méthode :
			\begin{itemize}
				\item Calcul des pseudo-valeurs Jackknife : $\hat{\theta_i}^* = n\hat{\theta} - (n-1) \hat{\theta_i}$
				\item Calcul de l'estimateur Jackknife : $\hat{\theta}^* = \frac{1}{n} \sum\limits_{i=1}^n \hat{\theta_i}^*$
			\end{itemize}
			\vspace{15px}
			Exemple : Estimation Jackknife de la variance sur matlab
		\end{frame}




	% ///////////////////////////////////// %
	% /// Évaluation de la stabilité ////// %
	\section{Application à l'évaluation de la stabilité}

		%%
		%%% Approche intuitive
		%%
		\subsection{Approche intuitive}
		\begin{frame}
			\frametitle{Approche intuitive}
			On enlève une observation à l'extrémité de l'ensemble des observations et on calcule la moyenne réduite (\textit{la moyenne sans prendre en compte cette observation}).

			\vspace{15px}
			La moyenne est trop fragile pour être stable si :
			\begin{itemize}
				\item la moyenne réduite est proportionnellement trop éloignée de la moyenne originale \textbf{OU} la moyenne réduite est en dehors d'un intervalle défini comme acceptable
				\item l'ensemble des observations est suffisamment petit pour que cet événement ait une probabilité suffisante de se produire
			\end{itemize}
		\end{frame}


		%%
		%%% Exemple : décision à propos de la stabilité
		%%
		\subsection{Exemple : décision à propos de la stabilité}
		\begin{frame}
			\frametitle{Exemple : décision à propos de la stabilité}
			Rappels des variables :
			\begin{itemize}
				\item $\hat{\theta} = 120.1 $
				\item $imr_{90 \%} = \intervalleff{112.21}{122.32} \Leftrightarrow P(\hat{\theta} \in \intervalleff{112.21}{122.32}) = 0.9$ (suppression de 62 et 400)
			\end{itemize}

			\vspace{10px}
			Suppression de la valeur la plus élevée (400) :
			\begin{itemize}
				\item $\hat{\theta}_{20} = 105.37 $ réduction de la moyenne de 12.3\%
				\item probabilité $P(\hat{\theta} = 105.37) = \frac{1}{20} = 5 \%$
			\end{itemize}

			\vspace{10px}
			Mais $imr_{90 \%} = \intervalleff{112.21}{122.32}$ indique que la probabilité que la moyenne réduite diminue à $112.21$ (réduction de 7\%) est de seulement 10\%. La borne supérieure $122.32$ est très proche de $\hat{\theta}$.\\

			\vspace{10px}
			On va se focaliser sur l'élimination de \textbf{la valeur extrême supérieure} (400).
		\end{frame}

		\subsection{Exemple : décision à propos de la stabilité (suite)}
		\begin{frame}
			\frametitle{Exemple : décision à propos de la stabilité (suite)}
			Récapitulatif :
			\begin{itemize}
				\item 5 \% de proba que la moyenne diminue de 120.1 à \textbf{105.37} (12 \% de changement)
				\item 10 \% de proba que la moyenne diminue de 120.1 à \textbf{112.21} (7 \% de changement)
			\end{itemize}

			\vspace{5px}
			Décision ?
			\begin{itemize}
				\item troublé par ces proportions ou ces probabilités $\Rightarrow$ on décide que la moyenne est \textbf{instable}
				\item ces changements paraissent vraisemblables $\Rightarrow$ on décide que la moyenne est \textbf{stable}
			\end{itemize}

			\vspace{5px}
			Avantages et inconvénients :
			\begin{itemize}
				\item Vision claire de ce qui arrive si on supprime un élément aux extrémités des observations
				\item calculs faciles et rapides
				\item pas de procédure de décision standardisée
			\end{itemize}
		\end{frame}
		
		
		\section{Autres applications}
		%%
		%%% Autres estimations
		%%
		\subsection{Autres estimations}
		\begin{frame}
			\frametitle{Autres estimations}
			\begin{itemize}
				\item Intervalle de confiance à $\alpha - 1$ : $I_{\alpha-1} = t_{\frac{\alpha}{2};n-1}\sqrt{\hat{\sigma_i}^2}$ avec $\hat{\sigma_i}^2(\hat{\theta}^*) = \frac{1}{n}\hat{\sigma_i}^2(\hat{\theta_i}^*) = \frac{1}{n(n-1)}\sum\limits_{i=1}^n (\hat{\theta}^* - \hat{\theta_i}^*)^2$
				\item Calcul de l'écart type : $ecartType_{jack} = \sqrt{\frac{(n - 1)}{n} \sum\limits_{i=1}^n ((\frac{1}{n} \sum\limits_{i=1}^n \hat{\theta_i}) - \hat{\theta_i})^2}$
				\item Test d'hypothèse $H_0 : \theta = \theta_0$
			\end{itemize}
		\end{frame}
		
		%%
		%%% Autres méthodes
		%%
		\subsection{Autres méthodes}
		\begin{frame}
			\frametitle{Autres méthodes}
			Limites du Jackknife :
			\begin{itemize}
				\item Inefficace pour travailler sur des variables peu stables, qui changent beaucoup pour un petit changement de données, telles que la médiane
			\end{itemize}
			\vspace{15px}
			Évolutions du Jackknife :
			\begin{itemize}
				\item Bootstrap : Même principe que le Jackknife mais utilisation d'un tirage aléatoire avec remise à partir de l'ensemble des données.
				\item Jackknife par bloc : Même principe mais on ignore un nombre fixé d'obeservations 
			\end{itemize}

		\end{frame}

% 		\begin{frame}
% 			\frametitle{Estimation Jackknife de l'erreur type}
% 			Si $\hat{\theta} = \overline{x}$ on a :\\
% 
% 			\vspace{5px}
% 			$\hat{\theta_i} = \frac{n\overline{x} - x_i}{n - 1}$ et $\frac{1}{n} \sum\limits_{i=1}^n \hat{\theta_i} = \overline{x}$\\
% 
% 			\vspace{15px}
% 			Ainsi :
% 			\[ecartType_{jack} = \sqrt{\sum\limits_{i=1}^n \frac{(x_i - \overline{x})^2}{n}} = \frac{\overline{\sigma}}{\sqrt{n}}\]
% 			où $\overline{\sigma}$ est l'écart type sans biais.
% 		\end{frame}


% Fin du document
\end{document}